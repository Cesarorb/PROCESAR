\documentclass[a4paper,12pt]{article}
\usepackage[utf8]{inputenc}
\usepackage{geometry}
\geometry{margin=1in}
\usepackage{parskip}
\usepackage[spanish]{babel}
\usepackage{csquotes}
\usepackage[style=apa]{biblatex} % Puedes seguir utilizando el estilo APA
\usepackage{url} % Paquete para manejar URLs correctamente

\title{Subcuentas que Integran la Cuenta Individual del Trabajador y sus Respectivas Aportaciones}
\author{}
\date{}

\begin{document}

\maketitle

En el sistema de seguridad social en México, la cuenta individual del trabajador es fundamental para el manejo y administración de las aportaciones y fondos destinados a su retiro. Esta cuenta se integra por diversas subcuentas, cada una con una función y propósito específicos. Las principales subcuentas que componen la cuenta individual del trabajador son la subcuenta de Retiro, la subcuenta de Vivienda, y la subcuenta de Ahorro Voluntario.

La \textbf{subcuenta de Retiro} es una de las más importantes, ya que está destinada a acumular los recursos para el retiro del trabajador. De acuerdo con el Instituto Mexicano del Seguro Social (IMSS, 2022), esta subcuenta recibe aportaciones del empleador equivalentes al 2\% del salario base de cotización del trabajador, además de las aportaciones adicionales que realiza el trabajador de forma voluntaria \parencite{IMSS2022}.

Por otro lado, la \textbf{subcuenta de Vivienda} tiene como objetivo facilitar el acceso a créditos para la adquisición, construcción o mejora de vivienda. Según el Fondo Nacional de la Vivienda para los Trabajadores (INFONAVIT, 2023), esta subcuenta recibe una aportación equivalente al 5\% del salario base de cotización del trabajador, la cual es aportada por el empleador \parencite{INFONAVIT2023}.

Finalmente, la \textbf{subcuenta de Ahorro Voluntario} permite a los trabajadores hacer aportaciones adicionales a su cuenta individual. Esta subcuenta está diseñada para ofrecer una mayor flexibilidad y aumentar el monto disponible para el retiro. Según el Instituto del Fondo Nacional de la Vivienda para los Trabajadores (INFONAVIT, 2022), las aportaciones realizadas a esta subcuenta pueden variar según la voluntad del trabajador y se suman a las aportaciones obligatorias y voluntarias realizadas por el trabajador \parencite{INFONAVIT2022}.

En conclusión, cada una de estas subcuentas juega un papel crucial en la administración de los fondos del trabajador, asegurando que estén disponibles para su retiro, así como para otros beneficios relacionados con la vivienda. La correcta comprensión y gestión de estas subcuentas son esenciales para el bienestar financiero del trabajador en el futuro.

\newline
\newline

\begin{thebibliography}{9}

\bibitem{IMSS2022}
Instituto Mexicano del Seguro Social.
\textit{Subcuenta de Retiro}.
2022.
URL: \url{http://www.imss.gob.mx/subcuenta-retiro}. Accedido: 2024-08-09.

\bibitem{INFONAVIT2023}
Fondo Nacional de la Vivienda para los Trabajadores.
\textit{Subcuenta de Vivienda}.
2023.
URL: \url{http://www.infonavit.org.mx/subcuenta-vivienda}. Accedido: 2024-08-09.

\bibitem{INFONAVIT2022}
Instituto del Fondo Nacional de la Vivienda para los Trabajadores.
\textit{Subcuenta de Ahorro Voluntario}.
2022.
URL: \url{http://www.infonavit.org.mx/subcuenta-ahorro-voluntario}. Accedido: 2024-08-09.

\end{thebibliography}

\end{document}
